\documentclass[a4paper,12pt]{article} 
\usepackage[T2A]{fontenc}			
\usepackage[utf8]{inputenc}			
\usepackage[english,russian]{babel}	
\usepackage{amsmath,amsfonts,amssymb,amsthm,mathtools} 
\usepackage{wasysym}
\usepackage{amsmath}

\author{конспект от TheLostDesu}
\title{Числовые последовательности}
\date{\today}


\begin{document}
\maketitle
\section{Числовая последовательность}
Задана числовая последовательность $\{a_n\}$, если каждому натуральному числу поставлено в соответствие вещественное число $a_n$
\subsection{Способы задания числовой последовательности}
1) Формула. Например $a_n = n^2 + 1$. Тогда $a_1 = 2, a_2 = 5, a_3 = 10$\\
2) С помощью рекурентных соотношений. Например последовательность Фибоначчи. $a_1 = 1$, $a_2 = 1$, для $n\geq 3: a_n = a_{n-1} + a_{n-2}$ \\
3) Описание последовательности(словами). Например $a_n$ - nное простое число.\\
\subsection{Предел числовой последовательности}
Рассмотрим последовательность $a_n$ член которой равен $\frac{1}{n}$. Тогда при $n$ стремящемся к бесконечности $a_n$ стремится к нулю. \\
$\lim\limits_{x\to 0} a_n=a$, если для любого $\epsilon > 0$ найдется некоторый номер N, что все числа последовательности отличаются от $a$ не больше чем на $\epsilon$.\\
$\forall$ - квантор общности. Читается как <<Для всех>>\\
$\exists$ - квантор существования. Читается как <<Существует>>\\
Тогда в кванторах определение предела выглядит, как\\
$\forall \epsilon > 0$ $\exists N \forall n \geq N \Rightarrow |x_n - a| < \epsilon.$\\
Утверждение 1. \\
Если предел $x_n$ равен $a$ при $n\Rightarrow \inf$, то на всем $\{x_n\}$ найдется конечное число точек не принадлежащих окрестности $a$.\\
Утверждение 2. \\ 
У одной последовательности может быть только один предел\footnote{Или не существовать ни одного}. \\
Доказательство: Пусть есть два предела: a1, a2. Тогда, без потерь общности a1 > a2. Если a1 > a2, то a1 - a2 - положительное число. Тогда пусть $\epsilon $ = $\frac{a1 - a2}{3}$. Тогда, окрестности не пересекаются. Но начиная с некоторого номера элементы должны начать попадать в обе окрестности, что невозможно. Следовательно у последовательности всего один предел.
\subsection{Ограниченная последовательность}
Последовательность $\{x_n\}$ ограничена, если $\exists m, M, \forall n \Rightarrow M \geq x_n \geq m$. \\
Если у последовательности есть конечный предел при $n \rightarrow \inf$, то она ограничена. \\
Тогда, возьмем $\epsilon = 1$. А значит, что существует $n$, такое что $a - 1 \leq x_n \leq a+1$.
Тогда $M = max(x_1, x_2, x_3..., x_n, a + 1)$, а $m = min(x_1, x_2, x_3..., x_n, a - 1)$ И последовательность ограничена.	
\end{document}