\documentclass[a4paper,12pt]{article} % тип документа
\usepackage[T2A]{fontenc}			% кодировка
\usepackage[utf8]{inputenc}			% кодировка исходного текста
\usepackage[english,russian]{babel}	% локализация и переносы
\usepackage{amsmath,amsfonts,amssymb,amsthm,mathtools} 
\usepackage{wasysym}


\author{конспект от TheLostDesu}
\title{Метод математической индукции}
\date{\today}


\begin{document}
\maketitle
\section{Определение}
Для того, чтобы доказать, что верны утверждения из набора $p_{1}, p_{2}, p_{3}, p_{4}...p_{n}$
достаточно доказать два факта:

1) $p_{1}$ - верное утверждение

2) из справедливости $p_{k}$ следует справедливость $p_{k+1}$.\\
Например: Доказать, что $1^2+2^2+3^2+4^2+...+n^2=\frac{n(n+1)(2n+1)}{6}$\\
1) При n = 1 \hspace{50pt} $1^2=1(2)(3)/6$ \hspace{50pt} $1=1$. Это верно.\footnote{Базис или база индукции}\\	
2) Пусть для $k$ наша формула верна. Тогда вот что происходит для $k+1$:\\
$1^2+2^2+3^2+4^2+...+k^2+(k+1)^2$\\
По предположению, что $1^2+2^2+3^2+4^2+...+k^2 = \frac{k(k+1)(2k+1)}{6}$\\
$1^2+2^2+3^2+4^2+...+k^2+(k+1)^2=\frac{k(k+1)(2k+1)}{6}+(k+1)^2$\\
Вынесем за скобки $(k+1)$\\
$\frac{k(k+1)(2k+1)}{6}+(k+1)^2=(k+1)(\frac{k(2k + 1)}{6} + 1)=\frac{(k+1)(2k+7k+6)}{6}=\frac{(k+1)(k+2)(2k+3)}{6}$\\
Что как раз является результатом подстановки $k+1$ в доказываемую формулу\footnote{Переход или шаг индукции}\\
Формула верна для $n = 1$. Если формула верна для $n=k$, то она верна и для $n=k+1$, тогда утверждение верно для любого $n\in\mathbb{N}$, что и требовалось доказать.

\section{Неравенство Бернулли}
Давайте докажем неравенство Бернулли методом математической индукции.\\
Оно гласит: при $n\in\mathbb{N}$ и $x\geq-1$\hspace{50pt}$(1+x)^n\geq 1+nx$\\
Доказательство:
1) При $n=1$\hspace{50pt}$1+x\geq1+x$\hspace{50pt}Это верно.
2) Пусть верно $(1+x)^k\geq1+kx$\\
Из $x\geq-1$ \hspace{10pt} $1+x\geq0$\\
Домножим обе части неравенства на $x+1$\\
$(1+x)^{k+1}\geq(1+kx)(1+x)$\\
$(1+x)^{k+1}\geq1+(k+1)x+kx^2$.\hspace{10pt} Так как $kx^2\geq0$\\
$1+(k+1)x+kx^2\geq1+(k+1)x$.\hspace{10pt}А это значит, что\\
$(1+kx)^{k+1}\geq1+(k+1)x$.\\
Значит формула верна для $k+1$\\
Формула верна для $n = 1$. Если формула верна для $n=k$, то она верна и для $n=k+1$, тогда утверждение верно для любого $n\in\mathbb{N}$, что и требовалось доказать.

\section{Биномиальный коэффицент}
При $n\geq 0$, и $0\geq k\geq n$ \hspace{10pt} $C_{n}^k = \frac{n!}{k!(n-k)!}$\footnote{! - знак факториала. n! = 1*2*3*...*n. 0! = 1}\\
Свойства $C_{n}^k$:

1) $C_{n}^0 = C_{n}^n = 1$

2) $C_{n}^k = C_{n}^{n-k}$

3) $C_{n}^k + C_n^{k-1} = C_{n+1}^k$\\
Так как п.3 не очевиден, докажем его.\\
$C_{n}^k+C_n^{k-1}=\frac{n!}{k!(n-k)!}+\frac{n!}{(k-1)!(n-k)!}=\frac{n!}{(k-1)!(n-k)!}\left(\frac{1}{k} + \frac{1}{n-k+1}\right)$\\
Что при приведении выражения в скобках к общему знаменателю даёт\\
$\frac{n!(n+1)}{(k-1)!(n-k)!k(n+1-k)}=\frac{(n+1)!}{k!(n+1-k)!}=C_{n+1}^k$\\
что и следовало доказать.\\[1cm]
Теорема:
$(1+x)^n=1+C_{n}^1x + C_{n}^2x^2 +... + C_{n}^kx^k + ... + C_{n}^nx^n$, что можно записать как
\[\sum_{k=0}^{n}C_{n}^kx^k\]
Докажем это методом математической индукции.\\
1) При n = 1\hspace{50pt}1 + x = 1 + x. \hspace{50pt} Это верно.\\
2) Пусть формула справедлива для $n=k$.\\
$(1+x)^{k+1}=(1+x)^k(1+x)=\left(1+C_{k}^1x+C_{k}^2x^2+...C_{k}^kx^k\right)(1+x)$\\
Раскроем скобки.\\
$1+C_{k}^1x+C_{k}^2x^2+...C_{k}^kx^k+$\\
$x+C_{k}^1x^2+C_{k}^2x^3+...C_{k}^kx^{k+1}$\\
А из п.3 про $C_{n}^k$ эта сумма равна:
$1 + C_{k}^1x + C_{k}^2x^2+...C_{k+1}^{k+1}x^{k+1}$.\footnote{Последняя $C_{k+1}^{k+1}$ появляется из того, что она равна еденице.}\\
Формула сходится с предполагаемой для k+1.\\
Формула верна для $n = 1$. Если формула верна для $n=k$, то она верна и для $n=k+1$, тогда утверждение верно для любого $n\in\mathbb{N}$, что и требовалось доказать.\\
А из этого следует, что:

\[(a+b)^n = b^n(1+\frac{a}{b})^n = b^n\sum^n_{k=0}C_{n}^k\frac{a^k}{b^k}=\sum_{k=0}^nC_{n}^k*a^k*b^{n-k}\]
И аналогично, если вместо b за скобки выносить a
\[\sum_{k=0}^nC_{n}^k*a^{n - k}*b^k\]
\end{document}
