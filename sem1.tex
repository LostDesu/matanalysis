\documentclass[a4paper,12pt]{article} 
\usepackage[T2A]{fontenc}			
\usepackage[utf8]{inputenc}			
\usepackage[english,russian]{babel}	
\usepackage{amsmath,amsfonts,amssymb,amsthm,mathtools} 
\usepackage{wasysym}
\usepackage{amsmath}
\everymath{\displaystyle}

\author{конспект от TheLostDesu}
\title{Семинар по пределам}
\date{\today}


\begin{document}
\maketitle
Вспомним, что предел $\lim_{n \to \inf} = a$, когда $\forall \varepsilon > 0 \exists N \forall n \geq N)\Rightarrow  |x_n - a|<\varepsilon$ 

Последовательность раходится тогда, когда не существует a, такого, что а - предел этой последовательности. $\forall a \exists \varepsilon > 0 \forall N \exists n \geq N \Rightarrow |x_n - a| \geq \varepsilon$

Например $\lim_{n \to \inf} \frac{2n - 1}{2n + 1} = 1$. Докажем это по определению:\\
Пусть есть $\varepsilon > 0$.  Тогда, возьмем N = $\left[ \frac{1}{\varepsilon} \right] + 1$. Тогда для любого $n > N$: $x_n - 1 \leq \varepsilon$. Следовательно предел действительно равен единице.

Пусть у нас есть $a \in R$, $k \in N$. Доказать, что $\lim_{n \to inf} \frac{a}{\sqrt[k]{n}} = 0$. Если $|\frac{a}{\sqrt[5]{n}}| < \varepsilon$, то $\frac{a^k}{n} < \varepsilon ^ k$. Но тогда $n > \frac{|a|^k}{\varepsilon ^ k}$. Возьмем следующее целое за этим число. Оно и будет N. Значит, по определению предел равен 0. Ч.Т.Д.

\section*{Вычисление пределов. Свойства}
Пусть есть две последовательности $x_n$ и $y_n$. А также известно, что предел первой последовательности равен $a$, а второй - $b$. \\
$\lim_{n \to \inf} \alpha * x_n = a * \alpha$\\
$\lim_{n \to \inf} (x_n + y_n) = a + b$\\
$\lim_{n \to \inf} (x_n * y_n) = a * b$\\
$\lim_{n \to \inf} (x_n / y_n) = a / b$, если b не равно 0.

Если есть три последовательности $x_n y_n z_n$, и $x_n \leq y_n \leq z_n$, то последовательность $y_n$ называется <<зажатой>>. 	Если при этом $x_n и z_n$ стремятся к одному числу, то и $y_n$ стремится к этому числу.

При нахождении предела частного обычно стоит выносить самое <<быстрорастущее>> слагаемое за скобки в числителе и знаменателе. Тогда получится пределы вида $\frac{n^k(const)}{n^j(const)}$. Эту дробь можно сократить, и определить предел данной последовательности.

Домашнее задание:
стр 137 номер 2(3-5), 26(1-6), 36(1-3)

\end{document}